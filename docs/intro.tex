\section{Introdução}
Em 1952 o computador MANIAC (Mathematical Analyzer, Numerical Integrator, and Computer) ou IAS foi construindo, prometendo resolver problemas matemáticos antes tidos como ou muito difíceis, ou muito demandantes de processamento humano. Em geral, podemos associar os seguintes problemas que rodavam no IAS (citados por Dyson em Turing’s Cathedral: The Origins of the Digital Universe \cite{dyson2012turing}) ao objeto de estudo do nosso projeto, a física:

\begin{enumerate}
\item Explosões nucleares, medidas em microsegundos
\item Ondas de choque, que variavam em tempo de microsegundos a minutos
\item Metereologia
\item Evolução biológia 
\item Evolução estelar
\end{enumerate}

Para explicitar esses avanços e, em especial, explorar a computação gráfica aplicada na física sobre o tempo, escolheu-se trabalhar com o tema de simulação de ambientes, unindo temáticas como sistemas de partículas, fluidos e terrenos. Estes tópicos, apesar de a priori parecem desconexos e um pouco restritivos quanto ao tema, são extremamente relevantes para a computação gráfica, e são amplamente utilizados em jogos, filmes e outras mídias visuais, além de terem fomentado a base para todas as simulações e aplicações físicas futuras.  

Faz-se uma menção honrosa aos corpos rígidos e tecidos, que também são temas extremamente relevantes e interessantes, mas que por limitações de tempo e escopo do projeto, não puderam ser abordados.
