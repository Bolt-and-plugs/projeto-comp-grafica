\section{Sistemas de Partículas}

O primeiro artigo referente a modelagem de eventos chamados "nebulosos" foi escrito e publicado no ano de 1983, por William T. Reeves\cite{reeves1983}. Este paper representa a primeira aparição de um sistema de partículas, alterando a representação de primitivas anterior, como polígonos com extremidades definidas, mas como nuvens de partículas primitivas que definem seu volume (futuramente a representação Point Cloud também seria utilizada). Esse volume de partículas também não é uma entidade por si só, cada partícula tem seu *lifetime* próprio com a passagem do tempo (esta modelada por uma série de oscilações estocásticas - randômicas | ou pseudo). 

Este projeto deriva de ideias prévias relacionadas aos videogames (provavelmente a primeira aplicação de cg em escala), como o grandessíssimo Evans e Sutherland Flight Simulator \footnote{\href{https://www.youtube.com/watch?v=6W-qb_jHRhA}{link para o video do Flight Simulator - Evanas e Sutherland}}, que, além de ser o projeto inventor dos frame buffers (lembrem do swap buffers do opengl), iniciou (primitivamente) o processamento de explosões com um sistema simples de colisão e a tentativa de renderizar esse elemento "fuzzy" (como o fogo \footnote{Esta técnica foi empregada no filme \href{https://www.youtube.com/watch?v=x8X44NRltMM}{Star Trek II: The Wrath f Khan} - minuto 1.39 }) mas sem a aleaotiriedade esperada.

Para renderizar esse sistema, o seguinte pipeline é seguido: (1) novas partículas são introduzidas no sistema, (2) cada partícula recebe seus atributos individualmente, (3) toda partícula que exista no sistema depois de passar do seu tempo de vida, são removidas; (4) as partículas restantes se movem baseadas nos seus atributos dinâmicos e (5) a imagem do sistema é criada no frame buffer.

Este paper tratou alguns parâmetros dos sistemas de partículas:
\begin{itemize}
\item numero de partículas geradas (densidade): $NPartsf = MeanPartsf + Rand() \times VarPartsf \rightarrow$ media de partículas que o sistema deve ter somado a um um valor randômico de -1 a 1 * a variância de partículas desejado. 
\item variância: o programador pode alterar a media através de alguma função qualquer, alterando o tamanho do volume.
\end{itemize}

Em geral, cada nova partícula possui: 

\begin{enumerate}
  \setlength{\itemsep}{0.05em}
\item Posição inicial
\item Velocidade e direção iniciais
\item Tamanho inicial
\item Cor inicial
\item Transparência inicial
\item Formato
\item Tempo de vida (em frames)
\end{enumerate}

\subsubsection{Problemas encontrados}
O artigo cita três problemas: (1) partículas não podem interagir com outras superfícies (2) só há interação de fato através de planos de projeção que limitam o crescimento das partículas. (3) toda partícula eh um emissor de luz, que adiciona luz para as partículas internas e externas, não sendo factível com a realidade. Para nosso sistema posteriormente discutido na seção \ref{sec:metodologia}, o principal problema desta abordagem esta em seu custo computacional, tema o qual também será abordado.
