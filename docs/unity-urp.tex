\section{Geração Volumétrica de Nuvens em tempo real com Unity URP}
Nesta seção, abordaremos a implementação prática de um sistema de geração volumétrica de nuvens em tempo real utilizando o Unity Universal Render Pipeline (URP), apoiando-nos nos conceitos teóricos previamente discutidos. 


\subsection{Objetivos}

Este projeto tem, portanto, um foco educacional ao se utilizar uma ferramenta comum como o Unity e mostrar a viabilidade de se implementar uma simulação complexa e otimizada com GPU em tempo real, conduzindo uma experimentação de técnicas distinas, consonantemente a um aprimoramento academico dos alunos aqui presentes. Ao se desenvolver um sistema de geração volumétrica de nuvens em tempo real, esperamos alcançar os seguintes objetivos específicos:

\begin{itemize}
    \item \textbf{Implementação de técnicas de geração volumétrica:} Desenvolver e implementar técnicas avançadas de geração volumétrica de nuvens, como noise functions (Perlin, Simplex), Worley noise, e outras abordagens baseadas em shaders.
    \item \textbf{Otimização para tempo real:} Adaptar e otimizar os algoritmos de geração volumétrica para garantir que possam ser executados em tempo real dentro do Unity URP, mantendo um equilíbrio entre qualidade visual e desempenho.
    \item \textbf{Integração com Unity URP:} Integrar o sistema de geração volumétrica de nuvens com o Unity Universal Render Pipeline, aproveitando suas capacidades de renderização e iluminação para melhorar a aparência das nuvens.
\end{itemize}

\section{Metodologia}
\label{sec:metodologia}

A concretização dos objetivos educacionais deste projeto foi alcançada através da implementação de um sistema de simulação e renderização em tempo real, utilizando o motor Unity sob a arquitetura do \textit{Universal Render Pipeline} (URP). Esta plataforma foi selecionada por sua capacidade de integrar rotinas de computação aceleradas por GPU (GPGPU), um requisito fundamental para processar as simulações complexas em tempo real.

O \textit{pipeline} metodológico foi dividido em duas etapas principais, ambas executadas inteiramente no processador gráfico. A primeira etapa, a "modelagem de forma", consiste na simulação da dinâmica dos fluidos. Para este fim, foram empregados \textit{Compute Shaders} para resolver numericamente as equações que governam o comportamento dos fluidos, como a equação de advecção-difusão, que modela a densidade volumétrica. Esta abordagem permitiu a implementação eficiente dos passos de advecção (pelo método semi-Lagrangiano), difusão e adição de fontes, atualizando um \textit{grid} tridimensional que representa o estado do volume a cada quadro. Podemos ver a interação dos componentes do projeto no seguinte 'diagrama': 

\begin{figure}[H]
\label{fig:fluid_pipeline}
\centering
\begin{tikzpicture}[
    node distance=10mm and 18mm,  % Distância horizontal aumentada
    box/.style = {draw, rounded corners, minimum width=40mm, minimum height=10mm, align=center, font=\footnotesize, fill=blue!5},
    cloudstyle/.style = {draw, cloud, cloud puffs=12, cloud ignores aspect, minimum width=36mm, minimum height=18mm, align=center, font=\footnotesize, fill=gray!10},
    small/.style = {draw, rounded corners, minimum height=6mm, align=center, font=\scriptsize, fill=white},
    arr/.style = {->, >=Stealth, line width=0.7pt},
    faded/.style = {draw=black!50, text=black!70, fill=white, rounded corners, font=\scriptsize}
]

% --- Pipeline Principal ---
\node (cpu) [small] {CPU / Unity Engine};

\node (sim) [box, below=of cpu, node distance=14mm]
    {Etapa 1: Simulação de Fluidos \\ (Compute Shader)};

\node (data) [box, below=of sim, node distance=25mm, fill=green!10]
    {Grid 3D (Textura Volumétrica)};

\node (render) [box, below=of data, node distance=25mm]
    {Etapa 2: Renderização \\ (Volumetric Ray Marching)};

\node (output) [cloudstyle, below=of render, node distance=25mm]
    {Imagem Final (Nuvem Volumétrica)};

% --- Passos da Simulação) ---
\node (step_title) [font=\tiny\bfseries, left=of sim, node distance=45mm] {Passos da Simulação:};
\node (step1) [faded, below=of step_title, node distance=6mm] {Adição de Fontes};
\node (step2) [faded, below=of step1] {Difusão};
\node (step3) [faded, below=of step2] {Advecção (Semi-Lagrangiana)};

% Fluxo Principal
\draw [arr] (cpu) -- (sim)
    node [midway, right, font=\tiny, xshift=2mm] {Dispatch (GPU)};
\draw [arr, line width=1pt] (sim) -- (data)
    node [midway, right, font=\scriptsize, xshift=1mm] {Atualiza Grid (Write)};
\draw [arr, line width=1pt] (data) -- (render)
    node [midway, right, font=\scriptsize, xshift=1mm] {Amostra Grid (Read)};
\draw [arr] (render) -- (output);

\draw [arr, line width=1pt, dashed] (data.east) to [bend right=50] (sim.east)
    node [midway, right, font=\scriptsize, xshift=1mm] {};

\coordinate (sim_input) at (sim.west);
\draw [arr, dashed, faded] (step1.east) -- (sim_input);
\draw [arr, dashed, faded] (step2.east) -- (sim_input);
\draw [arr, dashed, faded] (step3.east) -- (sim_input);

\end{tikzpicture}
\caption{Diagrama do Pipeline de Simulação e Renderização de Fluidos Volumétricos}
\end{figure}


Subsequentemente, a segunda etapa consistiu na renderização deste volume dinâmico. Foi implementada a técnica de \textit{volumetric ray marching}, cuja fundamentação teórica foi apresentada na Seção 5. Um \textit{shader} customizado foi desenvolvido para marchar raios através do \textit{grid} de densidade, que é continuamente atualizado pelo \textit{Compute Shader}. Este \textit{shader} de renderização foi integrado ao URP para amostrar o volume e calcular a interação da luz (absorção e espalhamento), gerando a representação visual final das nuvens volumétricas.


\section{Resultados}
\label{sec:resultados}

Resultados alcançados foram ok
