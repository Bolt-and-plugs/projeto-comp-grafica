\section{Metodologia}
\label{sec:metodologia}

A concretização dos objetivos educacionais deste projeto foi alcançada através da implementação de um sistema de simulação e renderização em tempo real, utilizando o motor Unity sob a arquitetura do \textit{Universal Render Pipeline} (URP). Esta plataforma foi selecionada por sua capacidade de integrar rotinas de computação aceleradas por GPU (GPGPU), um requisito fundamental para processar as simulações complexas em tempo real.

O \textit{pipeline} metodológico foi dividido em duas etapas principais, ambas executadas inteiramente no processador gráfico. A primeira etapa, a "modelagem de forma", consiste na simulação da dinâmica dos fluidos. Para este fim, foram empregados \textit{Compute Shaders} para resolver numericamente as equações que governam o comportamento dos fluidos, como a equação de advecção-difusão, que modela a densidade volumétrica. Esta abordagem permitiu a implementação eficiente dos passos de advecção (pelo método semi-Lagrangiano), difusão e adição de fontes, atualizando um \textit{grid} tridimensional que representa o estado do volume a cada quadro.

Subsequentemente, a segunda etapa consistiu na renderização deste volume dinâmico. Foi implementada a técnica de \textit{volumetric ray marching}, cuja fundamentação teórica foi apresentada na Seção 5. Um \textit{shader} customizado foi desenvolvido para marchar raios através do \textit{grid} de densidade, que é continuamente atualizado pelo \textit{Compute Shader}. Este \textit{shader} de renderização foi integrado ao URP para amostrar o volume e calcular a interação da luz (absorção e espalhamento), gerando a representação visual final das nuvens volumétricas.


